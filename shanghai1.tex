\documentclass{beamer}
\usepackage{amsfonts}
\usepackage{amsmath}
\usepackage{graphicx}
\usepackage{amssymb}
\usepackage{bm}
\usepackage{movie15}
\usepackage{color}
\usepackage{fancybox}
\usepackage{pgf,pgfarrows,pgfnodes,pgfautomata,pgfheaps,pgfshade}

\definecolor{dgreen}{rgb}{0.,0.6,0.}

\newcommand{\traitdroit}{
   \begin{pgfpicture}
      \pgfsetlinewidth{1.5pt}
      \pgfpathmoveto{\pgfxy(0,0)}
      \pgfpathlineto{\pgfxy(0.8,0)}
      \pgfusepath{stroke}
   \end{pgfpicture}
}
\newcommand{\flechedroite}{
   \begin{pgfpicture}
      \pgfsetlinewidth{0.8pt}
      \pgfsetarrowsend{latex}
      \pgfpathmoveto{\pgfxy(0,0)}
      \pgfpathlineto{\pgfxy(2.0,0)}
      \pgfusepath{stroke}
   \end{pgfpicture}
}
\newcommand{\flechediag}{
   \begin{pgfpicture}
      \pgfsetlinewidth{0.8pt}
      \pgfsetarrowsend{latex}
      \pgfpathmoveto{\pgfxy(0,0)}
      \pgfpathlineto{\pgfxy(2.0,-0.5)}
      \pgfusepath{stroke}
   \end{pgfpicture}
}

\newcommand{\bmu}{\bm{\mu}}

\mode<presentation>
{
   \usetheme{Boadilla}
}



\title[From first-principles to material properties]
{
Molten Salts:  Predicting physical and (electro)chemical properties from first-principles 
}
\author[M. Salanne]{
  M. Salanne}
\institute[UPMC]{
  Universit\'e Pierre et Marie Curie -- Paris -- France  \\
  http://www.pecsa.upmc.fr
} 
\date[Mar. 27, 2013]{\alert{Shanghai Institute of Applied Physics - Mar. 27, 2013}}

\begin{document}

\makeatletter
  \@ifundefined{inserttotalframenumbernew}{
    \gdef\inserttotalframenumbernew{1}
  }{}
  \gdef\inserttotalframenumber{\inserttotalframenumbernew}
\makeatother

\begin{frame}
 \titlepage
 \vspace{-0.5cm}
   \begin{figure}
   \includegraphics[height=3cm]{../../2010/tcsdm-decembre2010/snapshots/libef3-network}
   \end{figure}
\end{frame}

\begin{frame}
   \frametitle{Universit\'e Pierre et Marie Curie}
  \begin{figure}
    \includegraphics[width=5cm]{../../2010/seminar-niigata-mars2010/jussieu.jpg}\hspace{0.5cm}
    \includegraphics[width=5cm]{../../2013/shanghai-27032013/jardin_plantes.png}
  \end{figure}
  \begin{itemize}
      \item[$\bullet$] Located in the center of Paris
      \item[$\bullet$] Largest university in France 
      \item[$\bullet$] PECSA laboratory: 60 researchers, 45 Ph.D.s, 25 post-docs
      \item[$\bullet$] In the field of molten salts: 4 researchers, 1 Ph.D., 2 post-docs
  \end{itemize}
\end{frame}

\begin{frame}
   \frametitle{Molten salts in the generation IV nuclear industry}
   \begin{columns}
      \begin{column}{6cm}
         \begin{itemize}
            \item[$\bullet$] Molten Salt Fast Reactor:
 \newline \alert{fuel}, \alert{solvent} and \alert{coolant} 
            \item[$\bullet$] Very High Temperature Reactor: alternative to gas as a \alert{primary coolant}   
            \item[$\bullet$] Sodium-Cooled Fast Reactor: \alert{secondary coolant}
         \end{itemize}
      \end{column}
      \begin{column}{6cm}
         \begin{figure}
            \includegraphics[width=6.0cm]{../../2009/simades-120609/msr.jpg}
         \end{figure}
         \vspace{-0.5cm}
         \begin{center}
            {\small \it Molten Salt Reactor concept}
         \end{center}
      \end{column}
   \end{columns}
\end{frame}

\begin{frame}
   \frametitle{The EVOL project}
   EVOL: Evaluation and Viability of Liquid Fuel Fast Reactor Systems
         \vspace{0.5cm}
   \begin{itemize}
      \item[$\bullet$] Objective: propose a design of \alert{Molten Salt Fast Reactor} based on a LiF-ThF$_4$ fuel (no beryllium)
         \vspace{0.5cm}
      \item[$\bullet$] Founded by European commission (2009-2013)
         \vspace{0.5cm}
      \item[$\bullet$] 10 partners from France, Germany, UK, Italy, Belgium etc (leader: CNRS, France)
         \vspace{0.5cm}
      \item[$\bullet$] Our role in the project: model the physical and electrochemical properties of LiF-ThF$_4$
   \end{itemize}
\end{frame}

\begin{frame}
   \frametitle{Simulation method: molecular dynamics}
   \begin{columns}
      \begin{column}{6cm}
         \begin{itemize}
            \item[$\bullet$] Newton's equation of motion solved at each time step:\\
            \begin{equation}m^i\ddot{\vec{r}}^i=\sum_{j\ne i}\vec{F}^{j\rightarrow i}=-\frac{\partial V}{\partial \vec{r}^i}\nonumber\end{equation}\\
            where $V$ is the \alert{interaction potential}
  
            \item[$\bullet$] Trajectory of particles over a few nanoseconds
            \item[$\bullet$] Determination of \alert{structural}, \alert{thermodynamic} and \alert{transport} properties
         \end{itemize}
      \end{column}
      \begin{column}{6cm}
         \begin{figure}
     \includemovie[poster,text={Loading movie}]{6cm}{6cm}{../../2011/MS9/sio2na2o.mpg}
         \end{figure}
         \vspace{-0.5cm}
         \begin{center}
            {\small \it Simulation cell for SiO$_2$-Na$_2$O}
         \end{center}
      \end{column}
   \end{columns}
\end{frame}


\begin{frame}
  \frametitle{Outline}
  \tableofcontents
\end{frame}

\section{Polarizable Ion Model}
\begin{frame}
   \frametitle{Pairwise additive components}
   \begin{itemize}
      \item[$\bullet$] Charge-charge interactions:
            \[
            V_{\rm charge-charge}=\sum_{i, j > i}\frac{q^i q^j}{r^{ij}}
            \]
      \item[$\bullet$] Short-range repulsion:
            \[
            V_{\rm repulsion}=\sum_{i,j>i}B^{ij}\exp(-a^{ij}r^{ij})
            \]
      \item[$\bullet$] Dispersion interactions:
            \[
            V_{\rm dispersion}=-\sum_{i,j>i}\left[ f_6^{ij}(r^{ij})\frac{C_6^{ij}}{(r^{ij})^6}+f_8^{ij}(r^{ij})\frac{C_8^{ij}}{(r^{ij})^8}\right]
            \]
   \end{itemize}
\end{frame}
\begin{frame}
   \frametitle{Polarization component}
   \begin{itemize}
      \item[$\bullet$] Charge-dipole and dipole-dipole interactions
      \begin{eqnarray}
      V_{\rm polarization}&=& \sum_{i,j>i}\left( q^i \mu_\alpha^j g_D^{ij}(r^{ij})-q^j \mu_\alpha^i g_D^{ji}(r^{ij})\right) \nabla_\alpha \frac{1}{r^{ij}} \nonumber \\
                          & &\sum_{i,j>i} -\mu_\alpha^i \mu_\beta^j  \nabla_\alpha \nabla_\beta \frac{1}{r^{ij}}  \nonumber \\
                          & & +\sum_{i}\frac{1}{2\alpha^i}\mid \vec{\mu}^i \mid^2 \nonumber
      \end{eqnarray}
      where $ \{ \vec{\mu}^i \}_{i\in[1,N]} $ are the induced dipoles on each ion (\alert{additional degrees of freedom})
      \item[$\bullet$] Polarization component has a many-body character
      \item[$\bullet$] Parameterization from a first-principles DFT procedure (no experimental information)
   \end{itemize}
\end{frame}

\begin{frame}
   \frametitle{Why including polarization effects?}
   \begin{center}
   \includegraphics[width=6cm]{../../2010/tcsdm-decembre2010/polarizationeffect}
   \end{center}
   \begin{itemize}
      \item[$\bullet$] Without polarization: F$^{-}$ in the Be$^{2+}$-Be$^{2+}$ axis
      \item[$\bullet$] With polarization: Be-F-Be $\approx$ 136$^\circ$
   \end{itemize}
\end{frame}

\begin{frame}
   \frametitle{Which systems can we simulate?}
   Potentials obtained for:
      \vspace{0.3cm}
   \begin{itemize}
      \item[$\bullet$] Fluorides: LiF, NaF, KF, RbF, CsF, BeF$_2$, MgF$_2$, CaF$_2$, SrF$_2$, BaF$_2$, AlF$_3$, YF$_3$, LaF$_3$, ZrF$_4$, ThF$_4$
      \vspace{0.3cm}
      \item[$\bullet$] Chlorides: LiCl, NaCl, KCl, RbCl, CsCl, AlCl$_3$
      \vspace{0.3cm}
      \item[$\bullet$] Oxides: SiO$_2$, GeO$_2$, MgO, Li$_2$O, B$_2$O$_3$, Na$_2$O, Al$_2$O$_3$, ZrO$_2$, Sc$_2$O$_3$, Y$_2$O$_3$, CeO$_2$
      \vspace{0.3cm}
      \item[$\bullet$] Ionic liquids: 1-ethyl-3-methylimidazolium chloride (EMICl)
   \end{itemize}
\end{frame}


\begin{frame}
   \frametitle{Testing the potentials}
            \vspace{-0.29cm}
            \begin{center}
            \includegraphics[width=5.6cm]{/Users/salanne/Recherche/Redaction/Presentations/2011/cambridge-01062011/fig-validation/exafs-naf-zrf4} \includegraphics[width=5.6cm]{/Users/salanne/Recherche/Redaction/Presentations/2011/cambridge-01062011/fig-validation/diffusionF-lifkf} 
            \end{center}
   \vspace{-0.5cm}
   \begin{figure}
      \includegraphics[width=11cm]{../../2010/ifp-octobre2010/tab-woff}
   \end{figure}
%  \vspace{-0.5cm}
%  \begin{item[$\bullet$]ize}
%  \end{item[$\bullet$]ize}
{\scriptsize  Rotenberg, Salanne, Simon \& Vuilleumier, {\it Phys. Rev. Lett.} {\bf 104}, 138301 (2010)}
\end{frame}

\AtBeginSection[]
{
  \begin{frame}<beamer>
    \frametitle{Outline}
    \tableofcontents[current,currentsection]
  \end{frame}
}

\section{FLiBe mixtures}
\begin{frame}
   \frametitle{Structure of a ``simple'' molten salt: LiF}
RDFs in molten LiF @ 1123~K:
   \begin{center}
   \includegraphics[width=6cm]{../../2010/tcsdm-decembre2010/rdf/rdfLiF}
   \end{center}
Structure dominated by charge-ordering effects
\end{frame}

\begin{frame}
   \frametitle{Structure of an ``associated'' molten salt: BeF$_2$}
RDFs in molten BeF$_2$ @ 1300~K:
   \begin{center}
   \includegraphics[width=6cm]{../../2010/tcsdm-decembre2010/rdf/rdfBeF2resize}
   \end{center}
\begin{itemize}
\item[$\bullet$] Structure dominated by charge-ordering effects at long-range only 

\item[$\bullet$] Be-F: first peak very intense, first minima close to 0 
\item[$\bullet$] Be-Be and F-F not superposed at short-range
\end{itemize}

\end{frame}

\begin{frame}
   \frametitle{Structure of an ``associated'' molten salt: BeF$_2$}
   \begin{center}
   \includegraphics[width=10cm]{../../2010/tcsdm-decembre2010/snapshots/bef2-network}
   \end{center}
   Network of edge-sharing BeF$_2$ tetraedra: structural analog of SiO$_2$
\end{frame}

\begin{frame}
   \frametitle{X-ray diffraction}
   \begin{columns}
      \begin{column}{6.0cm}
      Pure LiF:
      \begin{center}
       \includegraphics[width=6.0cm]{../../2010/tcsdm-decembre2010/xray/xraylif-resize}
      \end{center}
      \end{column}
      \begin{column}{6.0cm}
      Pure BeF$_2$:
      \begin{center}
       \includegraphics[width=6.0cm]{../../2010/tcsdm-decembre2010/xray/xraybef2-resize}
      \end{center}
      \end{column}
   \end{columns}
   \begin{equation}
   H(k) = \frac{S_{\rm X-ray}(k)-(\sum_\alpha x_\alpha f_\alpha (k))^2}{(\sum_\alpha x_\alpha f_\alpha (k))^2} \nonumber
   \end{equation} 
   \vspace{0.5cm}  
   
   \scriptsize{Heaton, Brookes, Madden, Salanne, Simon \& Turq {\it J. Phys. Chem. B}, 110, 11454 (2006)}
\end{frame}



%\section{Mixing effects}
\begin{frame}
   \frametitle{Structure of LiF-BeF$_2$ mixtures}
   When adding LiF to BeF$_2$, a progressive depolymerization is observed:
   \only<1>{
   \begin{center}
   \includegraphics[width=10cm]{../../2010/tcsdm-decembre2010/snapshots/bef2-network}\\
   Molten BeF$_2$
   \end{center}
   }
   \only<2>{
   \begin{center}
   \includegraphics[width=10cm]{../../2010/tcsdm-decembre2010/snapshots/libef3-network}\\
   Molten LiBeF$_3$
   \end{center}
   }
   \only<3>{
   \begin{center}
   \includegraphics[width=10cm]{../../2010/tcsdm-decembre2010/snapshots/li3bef5-network}\\
   Molten Li$_3$BeF$_5$
   \end{center}
   }
\end{frame}

\begin{frame}
  \frametitle{Speciation in LiF-BeF$_2$ mixtures}
  \begin{center}
   \includegraphics[width=6cm]{../../2010/tcsdm-decembre2010/speciation}
   \end{center}
   \begin{itemize}
     \item[$\bullet$] Progressive depolymerization
     \item[$\bullet$] Diluted mixtures: Fluoroberyllate species (BeF$_4^{2-}$, Be$_2$F$_7^{3-}$, Be$_3$F$_{10}^{4-}$...)
   \end{itemize}
   \vspace{0.3cm}
   \scriptsize{Salanne, Simon, Turq, Heaton \& Madden, {\it J. Phys. Chem. B}, {\bf 110}, 11461 (2006)}
\end{frame}

%\begin{frame}
%  \frametitle{Chemical equilibrium}
%   \begin{itemize}
%     \item The polymerization reaction: 
%    \begin{equation}
%    {\rm Be}_n{\rm F}_{3n+1}^{(n+1)-} \hspace{0.2cm}+ \hspace{0.2cm}{\rm BeF}_4^{2-} \hspace{0.5cm}\leftrightharpoons \hspace{0.5cm} {\rm Be}_{n+1}{\rm F}_{3n+4}^{(n+2)-} \hspace{0.2cm}+ \hspace{0.2cm}{\rm F}^- \nonumber
%    \end{equation}
%      
%     \item can also be written: 
%     \begin{equation}
%       2 \hspace{0.2cm}\equiv{\rm Be}-{\rm F}^{-1/2} \hspace{0.5cm}\leftrightharpoons \hspace{0.5cm}\equiv{\rm Be}-{\rm F}-{\rm Be}\equiv \hspace{0.2cm}+ \hspace{0.2cm}{\rm F}^- \nonumber
%     \end{equation}
%   \end{itemize}
%   Reaction mechanism based on ${\rm Be}-{\rm F}$ bonds formations / breakings
%\end{frame}
%
%\begin{frame}
%  \frametitle{Possible reaction mechanism 1}
%  Formation of a 3-fold coordinated Be
%       \begin{equation}
%          \equiv{\rm Be}-{\rm F}^{-1/2} \hspace{0.5cm} \rightarrow \hspace{0.5cm} \equiv{\rm Be}^{+1/2} \hspace{0.2cm}+ \hspace{0.2cm}{\rm F}^-   \nonumber
%       \end{equation}
%  Which reacts with another complex to form a Be-F-Be bridge
%       \begin{equation}
%          \equiv{\rm Be}^{+1/2} \hspace{0.2cm} +\hspace{0.2cm} \equiv{\rm Be}-{\rm F}^{-1/2} \hspace{0.5cm} \rightarrow \hspace{0.5cm} \equiv{\rm Be}-{\rm F}-{\rm Be}\equiv \nonumber
%       \end{equation}
%\end{frame}
%
%\begin{frame}
%  \frametitle{Possible reaction mechanism 2}
%  
%  Formation of a complex with a Be-F-Be bridge containing a 5-fold coordinated Be
%       \begin{eqnarray}
%          \equiv{\rm Be}-{\rm F}^{-1/2} \hspace{0.2cm} +\hspace{0.2cm} \equiv{\rm Be}-{\rm F}^{-1/2} \hspace{0.5cm} \rightarrow \hspace{0.5cm} \equiv{\rm Be}-{\rm F}-&{\rm Be}&\equiv \nonumber \\
% &|& \nonumber \\   
% &{\rm F}^-& \nonumber    
%       \end{eqnarray}
%Followed by the elimination of the excess F  
%       \begin{eqnarray}
%          \equiv{\rm Be}-{\rm F}-&{\rm Be}&\equiv  \hspace{0.5cm} \rightarrow \hspace{0.5cm}  \equiv{\rm Be}-{\rm F}-{\rm Be}\equiv \hspace{0.2cm} +\hspace{0.2cm} {\rm F}^- \nonumber \\
% &|& \nonumber \\   
% &{\rm F}^-& \nonumber    
%       \end{eqnarray}
%\end{frame}

%\begin{frame}
%  \frametitle{Possible reaction mechanism 3}
%  
%  Concerted mechanism: Simultaneous formation of a Be-F-Be bridge and breaking of a terminal Be-F bond:
%     \begin{equation}
%        \hspace{0.2cm}\equiv{\rm Be}-{\rm F}^{-1/2}\hspace{0.2cm}+ \hspace{0.2cm}\hspace{0.2cm}\equiv{\rm Be}-{\rm F}^{-1/2} \hspace{0.5cm}\rightarrow \hspace{0.5cm}\equiv{\rm Be}-{\rm F}-{\rm Be}\equiv \hspace{0.2cm}+ \hspace{0.2cm}{\rm F}^- \nonumber
%     \end{equation}
%  In this mechanism the two Be atoms remain 4-fold coordinated during the reaction
%\end{frame}

\begin{frame}
  \frametitle{Observation of a reaction event}
  Evolution of the distance between one Be and two F atoms:
  \begin{center}
   \includegraphics[width=6cm]{../../2010/tcsdm-decembre2010/distanceBe-Fresize}
  \end{center}  
  Concerted mechanism: Simultaneous formation of a Be-F-Be bridge and breaking of a terminal Be-F bond:
     \begin{equation}
        \hspace{0.2cm}\equiv{\rm Be}-{\rm F}^{-1/2}\hspace{0.2cm}+ \hspace{0.2cm}\hspace{0.2cm}\equiv{\rm Be}-{\rm F}^{-1/2} \hspace{0.5cm}\rightarrow \hspace{0.5cm}\equiv{\rm Be}-{\rm F}-{\rm Be}\equiv \hspace{0.2cm}+ \hspace{0.2cm}{\rm F}^- \nonumber
     \end{equation}
  The two Be atoms remain 4-fold coordinated during the reaction
\end{frame}

\begin{frame}
   \frametitle{Infrared spectroscopy}
   Pure BeF$_2$: 
      \begin{center}
       \includegraphics[width=8.0cm]{../../2010/tcsdm-decembre2010/ir/bef2_irspec-resize}
      \end{center}
     \begin{itemize}
       \item[$\bullet$] Possible to separate the charge-charge (JJ), charge-dipole (MJ) and dipole-dipole (MM) correlations
       \item[$\bullet$] Important cancellation between JJ and MJ
     \end{itemize}
\end{frame}

\begin{frame}
   \frametitle{Viscosity}
   \includegraphics[width=6.0cm]{/Users/salanne/Recherche/Redaction/Presentations/2011/cambridge-01062011/viscobef2/visco-compo-resize}\uncover<2-3>{\includegraphics[width=6.1cm]{/Users/salanne/Recherche/Redaction/Presentations/2011/cambridge-01062011/viscobef2/visco-polym}}
   \begin{itemize}
 \item[$\bullet$] Viscosity raises suddenly with composition
 \item[$\bullet$] Link with the polymerization reaction?
 \item[$\bullet$]<3> Not observed for the electrical conductivity: why?
   \end{itemize}
\end{frame}
\begin{frame}
   \frametitle{Diffusion coefficients}
    \begin{figure}
       \includegraphics[width=8cm]{../../2008/simades-010208/coefdiff}
       \vspace{-0.5cm}
       \begin{center}
          {\small Diffusion coefficients}
       \end{center}
    \end{figure}
    \vspace{-0.5cm}
    \begin{itemize}
       \item[$\bullet$] Li$^+$ ions diffuse much faster.
       \item[$\bullet$] Similarities with alkali-silicates glasses and melts
    \end{itemize}
\end{frame}
\begin{frame}
   \frametitle{Preferential pathways for Li$^+$ ions diffusion}
    \begin{figure}
       \includegraphics[width=5cm]{../../2008/simades-010208/800K-6fois}
       \vspace{-0.5cm}
       \begin{center}
          {\small 80\%~BeF$_2$ mixtures - 800~K}
       \end{center}
    \end{figure}
    \vspace{-0.5cm}
    \begin{itemize}
       \item[$\bullet$] High Li$^+$ ions concentration zones.
       \item[$\bullet$] Cooperative motion of Li$^+$ ions.
    \end{itemize}
    \begin{center}
   \scriptsize{Salanne, Simon, Turq \& Madden, {\it J. Phys. Chem. B}, 111, 4678 (2007)}
    \end{center}
\end{frame}

\section{LiF-ThF$_4$ eutectic}

\begin{frame}
   \frametitle{Thermodynamic properties}
   \begin{columns}
      \begin{column}{6cm}
      Density:
      \end{column}
      \begin{column}{6cm}
      Surface tension:
      \end{column}
   \end{columns}
   \begin{columns}
      \begin{column}{6cm}
       \includegraphics[width=5cm]{../../../Articles/lif-thf4/figures/density_resize}
      \begin{equation}
        \rho= 5.16 -0.00105\times T  \nonumber
      \end{equation}
      Heat capacity:\\
      \begin{center}
        $C_p$ = 1.049 J g$^{-1}$ K$^{-1}$  
      \end{center}
      \end{column}
      \begin{column}{6cm}
      \begin{center}
       \includegraphics[width=4.5cm]{../../2013/shanghai-27032013/lvinterface}

        $\gamma$ = 196.3 mN m$^{-1}$
      \end{center}
      
      \end{column}
   \end{columns}
    
    \vspace{0.2cm}
    \begin{center}
   \scriptsize{Dewan, Simon, Madden, Hobbs \& Salanne, {\it J. Nucl. Mater.}, 434, 322 (2013)}
    \end{center}
   
\end{frame}

\begin{frame}
   \frametitle{Transport properties}
   \begin{columns}
      \begin{column}{6cm}
      Viscosity:
      \end{column}
      \begin{column}{6cm}
      Diffusion:
      \end{column}
   \end{columns}
   \begin{columns}
      \begin{column}{6cm}
      \begin{center}
       \includegraphics[width=5cm]{../../../Articles/lif-thf4/figures/visco_resize}
      \end{center}
      \end{column}
      \begin{column}{6cm}
      \begin{center}
       \includegraphics[width=5cm]{../../../Articles/lif-thf4/figures/diff_resize}
      \end{center}
      \end{column}
   \end{columns}

    \vspace{0.5cm}
    Rather low viscosity, high density, high capacity $\rightarrow$ \alert{good coolant for a nuclear reactor} 
    \vspace{0.2cm}
    \begin{center}
   \scriptsize{Dewan, Simon, Madden, Hobbs \& Salanne, {\it J. Nucl. Mater.}, 434, 322 (2013)}
    \end{center}
   
\end{frame}

\section{Activity coefficients}
\begin{frame}
   \frametitle{Grouped extraction of actinides in molten salts}
   \begin{columns}
      \begin{column}{6cm}
         \begin{figure}
            \includegraphics[width=6cm]{../../2008/atelierparis-210108/process}
         \end{figure}
      \end{column}
      \begin{column}{6cm}
         \begin{itemize}
            \item[$\bullet$] U, Pu, minor actinides, lanthanides and fission products dissolved in the melt.
            \item[$\bullet$] Electrodeposition on a metallic cathode:
            \begin{itemize}
               \item solid: reactive (Ni) or not (W, Mo...)
               \item liquid (Bi, Cd, Al...)
            \end{itemize}
         \end{itemize}
      \end{column}
   \end{columns}
\vspace{1cm}
Examples of molten salt solvents: LiCl-KCl, LiF-AlF$_3$, LiF-CaF$_2$...
\end{frame}

\begin{frame}
   \frametitle{Electrochemical reactions}
   \begin{itemize}
      \item[$\bullet$] When the reduced species is deposited on the cathode:
          \begin{equation}
             {\rm M}_{\rm molten salt}^{n+}+n{\rm e}^- \rightarrow {\rm M}_{\rm metal} \nonumber
          \end{equation}
      \item[$\bullet$] Case of an inert solid cathode:
   \end{itemize}
             \begin{eqnarray}
                E&=&\alert{E^0_{{\rm M}^{n+}/{\rm M}}}+\frac{RT}{nF}\ln(\alert{\gamma_{{\rm M}^{n+}}})+\frac{RT}{nF}\ln(x_{{\rm M}^{n+}})\nonumber \\
&=&\alert{E^{0'}_{{\rm M}^{n+}/{\rm M}}}+\frac{RT}{nF}\ln(x_{{\rm M}^{n+}}) \nonumber
             \end{eqnarray}
   \begin{itemize}
      \item[$\bullet$] Reference state: pure MX$_n$ (liquid or supercooled depending on the temperature)
   \end{itemize}
\end{frame}

\begin{frame}
   \frametitle{Separation of two species}
   \begin{itemize}
      \item[$\bullet$] Efficacy of the separation of two metals M$^{3+}$ et N$^{3+}$ at the same initial concentration:
   \end{itemize}
   \begin{eqnarray}
     \eta&=&1-\exp\left[-\frac{3F(E^{0'}_{{\rm M}^{n+}/{\rm M}}-E^{0'}_{{\rm N}^{n+}/{\rm N}})}{RT} \right] \nonumber \\
 &=&1-\frac{\gamma_{{\rm N}^{3+}}}{\gamma_{{\rm M}^{3+}}}\exp\left[-\frac{3F(E^0_{{\rm M}^{n+}/{\rm M}}-E^0_{{\rm N}^{n+}/{\rm N}})}{RT} \right] \nonumber
   \end{eqnarray}
   \begin{itemize}
      \item[$\bullet$] $\eta>0.999$ if $\Delta E^{0'}>$0.149~V
      \item[$\bullet$] If $\Delta E^{0}$ is small, the $\gamma_{{\rm N}^{3+}}/ \gamma_{{\rm M}^{3+}}$ ratio may allow one to perform the separation
   \end{itemize}
   \alert{Calculation of activity coefficients by molecular dynamics?}
\end{frame}

\begin{frame}
   \frametitle{Calculation of activity coefficients: principle}
   \begin{center}
      \begin{tabular}{c c c}
      M$^{3+}$ (pure MCl$_3$) & $\xrightarrow{ RT\ln(\gamma_{{\rm M}^{3+}})}$ & M$^{3+}$ (LiCl-KCl) \\
\\
      \only<2>{\color{red} $\Delta G_1 \downarrow$} & & \only<2>{\color{red} $\downarrow \Delta G_2$} \\
\\
      N$^{3+}$ (pure NCl$_3$) & $\xrightarrow{ RT\ln(\gamma_{{\rm N}^{3+}})}$ & N$^{3+}$ (LiCl-KCl)
      \end{tabular}
   \end{center}
   \uncover<2>{
   Calculation of the Gibb's free energies for the transmutations in pure MCl$_3$ ({\color{red}$\Delta G_1$})
   and in LiCl-KCl solvent ({\color{red}$\Delta G_2$})

   \[{\color{red}\Delta G_2 - \Delta G_1 = RT\ln(\frac{\gamma_{{\rm N}^{3+}}}{\gamma_{{\rm M}^{3+}}})  }\]
   $\rightarrow$ Only the activity coefficient ratios are accessible. 
   }
\end{frame}
\begin{frame}
   \frametitle{Thermodynamic integration}
   \begin{columns}
      \begin{column}{6cm}
         \begin{figure}
            \includegraphics[width=6cm]{../../2008/atelierparis-210108/transmut}
         \end{figure}
            \vspace{-1.2cm}
         \uncover<2>{
            \begin{figure}
               \includegraphics[width=6cm]{../../2008/atelierparis-210108/integrationthermo}
            \end{figure}
         }
      \end{column}
      \begin{column}{6cm}
         \begin{itemize}
            \vspace{0cm}
            \item[$\bullet$] Progressive sliding from one cation to another using a parameter $\lambda$ ($0<\lambda<1$).

                  \alert{Interaction potential: $V(\lambda)=\lambda V_{{\rm M}^{3+}}+(1-\lambda) V_{{\rm N}^{3+}}$}
            \vspace{1cm}
            \item[$\bullet$] \uncover<2>{
            Calculation of
            \[ \Delta G_i = \int_0^1 \langle \frac{\partial V(\lambda)}{\partial \lambda} \rangle_\lambda {\rm d}\lambda
            \]
            }
         \end{itemize}
      \end{column}
   \end{columns}
\end{frame}
\begin{frame}
   \frametitle{Importance: grouped extraction of actinides in LiCl-KCl}
   \begin{center}
      \begin{tabular}{c c c  c c }
         & &   &   \multicolumn{2}{c}{\uncover<2>{\alert{in LiCl-KCl}}}  \\
         &$E^0$ & \hspace{2.5cm}  &  \multicolumn{2}{c} {\uncover<2>{$E^0+ \frac{RT}{3F}\ln(\gamma_{{\rm M}^{3+}})$}}  \\
         & &   &  &  \\
         0,65~V & \traitdroit &\uncover<2>\flechedroite &\uncover<2>\traitdroit & \only<2>{\alert{0,65~V}} \\%& \uncover<2-3>{(0,65~V)} \\ 
          & &U$^{3+}$ & & \\
 	 0,51~V & \traitdroit & & & \\
         & & \uncover<2>\flechediag & & \\
         & &Sc$^{3+}$  &   \uncover<2>\traitdroit & \only<2>{\alert{0,37~V}}\\%& \uncover<2-3>{(0,30~V)} \\
         & & & & \\
         0,23~V&   \traitdroit & Tb$^{3+}$ & & \\
         0,20~V&   \traitdroit &   \uncover<2>\flechediag &   \uncover<2>\traitdroit& \only<2>{\alert{0,15~V}} \\%& \uncover<2-3>{(0,14~V)} \\
         & &   \uncover<2>\flechediag  &   \uncover<2>\traitdroit & \only<2>{\alert{0,10~V}} \\%& \uncover<2-3>{(0,02~V)} \\
         & & Y$^{3+}$ &  & \\
         0,00~V&   \traitdroit &   \uncover<2>\flechedroite &  \uncover<2>\traitdroit & \only<2>{\alert{0,00~V}} \\%& \uncover<2-3>{(0,00~V)} \\
          & &La$^{3+}$  & & \\
      \end{tabular}
   \end{center}

\end{frame}
\begin{frame}
   \frametitle{Link between the activity coefficients anf the solvation shell}
   \begin{columns}
      \begin{column}{6cm}
         \vspace{-0.5cm}
         \begin{figure}
            \includegraphics[width=6cm]{../../2008/atelierparis-210108/rdf-likcl}
         \end{figure}
         \uncover<2>{
         \vspace{-1cm}
         \begin{figure}
            \includegraphics[width=6cm]{../../2008/atelierparis-210108/rayon}
         \end{figure}
         }
      \end{column}
      \begin{column}{6cm}
         \begin{itemize}
            \item[$\bullet$] Coordination numbers in LiCl-KCl:
                  \begin{itemize}
                     \item[$\bullet$] 6 for Sc$^{3+}$, Tb$^{3+}$, Y$^{3+}$
                     \item[$\bullet$] 6 or 7 for La$^{3+}$, U$^{3+}$
                  \end{itemize}
            \vspace{2cm}
            \item[$\bullet$]<2> Correlation between $\ln(\gamma_{{\rm M}^{3+}})$ and the ionic radius of M$^{3+}$
         \end{itemize}
            \vspace{2cm}
   \scriptsize{Salanne {\it et al.}, {\it J. Phys. Chem. B}, {\bf 112}, 1177 (2008)}
      \end{column}
   \end{columns}
\end{frame}
\begin{frame}
   \frametitle{Link between the activity coefficients anf the solvation shell}
   \begin{columns}
      \begin{column}{6cm}
         \vspace{-0.5cm}
         \begin{figure}
            \includegraphics[width=6cm]{../../2008/atelierparis-210108/rdf-likcl}
         \end{figure}
         \vspace{-1cm}
         \begin{figure}
            \includegraphics[width=5cm]{../../2012/habilitation/fukasawa}
         \end{figure}
      \end{column}
      \begin{column}{6cm}
         \begin{itemize}
            \item[$\bullet$] Coordination numbers in LiCl-KCl:
                  \begin{itemize}
                     \item[$\bullet$] 6 for Sc$^{3+}$, Tb$^{3+}$, Y$^{3+}$
                     \item[$\bullet$] 6 or 7 for La$^{3+}$, U$^{3+}$
                  \end{itemize}
            \vspace{2cm}
            \item[$\bullet$]<2> Correlation between $\ln(\gamma_{{\rm M}^{3+}})$ and the ionic radius of M$^{3+}$
         \end{itemize}
            \vspace{2cm}
   {\scriptsize Fukusawa {\it et al.}, {\it J. Nucl. Mater.} {\bf 424}, 17 (2012)}
      \end{column}
   \end{columns}
\end{frame}

\section{Conclusion \& Perspectives}

\begin{frame}
   \frametitle{Conclusion}
   \begin{itemize}
      \item[$\bullet$] Interaction potentials including many-body polarization effects for a series of molten fluorides
      \item[$\bullet$] Parameterization from first-principles calculations
      \item[$\bullet$] Access to the reactivity and the speciation in these media
      \item[$\bullet$] Prediction of transport properties
   \end{itemize}
   Perspectives:
   \begin{itemize}
      \item[$\bullet$] Development of interaction potentials for other elements (molten fluorides)  
      \item[$\bullet$] Activity coefficients in molten fluorides 
   \end{itemize}
\end{frame}

\begin{frame}
   \frametitle{Acknowledgements}
   \begin{itemize}
      \item[$\bullet$] Leslie Dewan (Massachusetts Institute of Technology, USA)
      \vspace{0.5cm}
      \item[$\bullet$] Anne-Laure Rollet, Christian Simon, Henri Groult, Pierre Turq (PECSA)
      \vspace{0.5cm}
      \item[$\bullet$] Catherine Bessada, Didier Zanghi, Olivier Pauvert (CEMHTI Orl\'eans, France)
      \vspace{0.5cm}
      \item[$\bullet$] Paul Madden (University of Oxford, UK)
   \end{itemize}
\end{frame}


\appendix
\makeatletter
  \immediate\write\@mainaux{\string\gdef\string\inserttotalframenumbernew{\insertframenumber}}
\makeatother





\end{document}
